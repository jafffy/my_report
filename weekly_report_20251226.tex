\documentclass[aspectratio=169]{beamer}

% Theme
\usetheme{Madrid}
\usecolortheme{whale}

% Packages
\usepackage{kotex}
\usepackage{graphicx}
\usepackage{booktabs}
\usepackage{xcolor}
\usepackage{colortbl}
\usepackage{tikz}
\usetikzlibrary{shapes.geometric, arrows, positioning}

% Colors
\definecolor{completed}{RGB}{34, 139, 34}
\definecolor{inprogress}{RGB}{70, 130, 180}
\definecolor{hold}{RGB}{255, 140, 0}
\definecolor{highlight}{RGB}{128, 0, 128}

% Remove navigation symbols
\setbeamertemplate{navigation symbols}{}

% Footer
\setbeamertemplate{footline}{
  \leavevmode%
  \hbox{%
    \begin{beamercolorbox}[wd=.333333\paperwidth,ht=2.25ex,dp=1ex,center]{author in head/foot}%
      \usebeamerfont{author in head/foot}최재원
    \end{beamercolorbox}%
    \begin{beamercolorbox}[wd=.333333\paperwidth,ht=2.25ex,dp=1ex,center]{title in head/foot}%
      \usebeamerfont{title in head/foot}주간 업무 보고
    \end{beamercolorbox}%
    \begin{beamercolorbox}[wd=.333333\paperwidth,ht=2.25ex,dp=1ex,right]{date in head/foot}%
      \usebeamerfont{date in head/foot}2025.12.26\hspace*{2em}
      \insertframenumber{} / \inserttotalframenumber\hspace*{2ex}
    \end{beamercolorbox}}%
  \vskip0pt%
}

% Title information
\title[주간 보고]{주간 업무 보고}
\author{최재원}
\date{2025년 12월 26일}
\institute{서울대학교병원 융합의학과}

\begin{document}

% ============================================
% Title Slide
% ============================================
\begin{frame}
  \titlepage
\end{frame}

% ============================================
% Overview Slide
% ============================================
\begin{frame}{업무 현황 Overview}
  \begin{table}
    \small
    \begin{tabular}{p{10cm} l}
      \toprule
      \textbf{프로젝트} & \textbf{상태} \\
      \midrule
      의료 프로그램 업무 자동화 매크로 개발 & \textcolor{hold}{\textbf{Hold}} \\
      SNOMED CT 기반 간호 용어 분류 모델 개발 & \textcolor{completed}{논문 초안 전달 완료} \\
      SNOMED CT \& LOINC mapping (Human-AI) & \textcolor{inprogress}{연구 계획 중} \\
      \textbf{DEEP-ICU: Synthetic Data 생성} & \textcolor{highlight}{\textbf{평가 프레임워크 수립}} \\
      사망 데이터 활용 연구 & \textcolor{completed}{데이터 추출 완료} \\
      뇌조직 생검술 후 뇌출혈 위험 예측 모델 & \textcolor{completed}{논문 초안 전달 완료} \\
      ARDS 연구 (김성민 교수님) & \textcolor{completed}{데이터 전달 완료} \\
      자폐 플랫폼 (김동영 연구원) & \textcolor{inprogress}{논문 초안 writing 중} \\
      \bottomrule
    \end{tabular}
  \end{table}
\end{frame}

% ============================================
% DEEP-ICU Title
% ============================================
\begin{frame}
  \begin{center}
    \vspace{2cm}
    {\Huge \textbf{DEEP-ICU}}

    \vspace{0.5cm}

    {\large Domain-Embedded Encodings for\\Personalized ICU outcomes}

    \vspace{1cm}

    {\color{gray} K-MIMIC Synthetic Data Generation}
  \end{center}
\end{frame}

% ============================================
% DEEP-ICU Goal
% ============================================
\begin{frame}{DEEP-ICU 목표}
  \begin{center}
    \begin{beamercolorbox}[sep=1em,wd=0.9\textwidth,center,rounded=true,shadow=true]{block body}
      \Large \textbf{K-MIMIC의 Synthetic Data를 생성}하여\\
      데이터 부족 문제 해결 및 프라이버시 보호
    \end{beamercolorbox}
  \end{center}

  \vspace{1cm}

  \textbf{활용 목적:}
  \begin{itemize}
    \item 데이터 증강 (Data Augmentation)
    \item 프라이버시 보호 데이터 공유
    \item Downstream Task 성능 검증
  \end{itemize}
\end{frame}

% ============================================
% Pipeline Overview
% ============================================
\begin{frame}{Synthetic Data 생성 파이프라인}
  \begin{center}
    \begin{tikzpicture}[
      node distance=1.2cm,
      box/.style={rectangle, rounded corners, minimum width=10cm, minimum height=1.4cm, text centered, draw=black, font=\small, text width=9.5cm, align=center},
      arrow/.style={thick,->,>=stealth}
    ]

    \node (step1) [box, fill=green!20] {
      \textbf{Step 1: ICD Code 생성} \\[2pt]
      K-MIMIC → \textbf{Halo 모델} → 그럴듯한 ICD codes {\small\color{gray}(이한주 선생님)}
    };

    \node (step2) [box, fill=orange!20, below=of step1] {
      \textbf{Step 2: Clinical Events 생성} \\[2pt]
      ICD codes → \textbf{Claude Sonnet 4.5} → CHARTEVENTS + LABEVENTS
    };

    \node (step3) [box, fill=blue!20, below=of step2] {
      \textbf{Step 3: 체계적 검증} \\[2pt]
      Fidelity / Utility / Privacy / Clinical Validity 평가
    };

    \draw [arrow] (step1) -- (step2);
    \draw [arrow] (step2) -- (step3);

    \end{tikzpicture}
  \end{center}
\end{frame}

% ============================================
% Claude Agent SDK
% ============================================
\begin{frame}{Step 2: Claude Agent SDK 활용}
  \small
  \textbf{Claude Agent SDK}를 사용하여 자율적 데이터 생성

  \vspace{0.3cm}

  \begin{columns}[T]
    \begin{column}{0.48\textwidth}
      \textbf{생성 대상 (CHARTEVENTS)}
      \footnotesize
      \begin{itemize}
        \setlength{\itemsep}{0pt}
        \item Capillary refill rate
        \item Blood pressure (Sys/Dia/Mean)
        \item GCS (Eye/Motor/Verbal/Total)
        \item Heart Rate, Respiratory rate
        \item SpO2, Temperature
        \item Glucose, pH
        \item Height, Weight, FiO2
      \end{itemize}
    \end{column}

    \begin{column}{0.48\textwidth}
      \textbf{생성 대상 (LABEVENTS)}
      \footnotesize
      \begin{itemize}
        \setlength{\itemsep}{0pt}
        \item Glucose (50809/50931)
        \item Oxygen saturation (50817)
        \item pH (50820)
      \end{itemize}

      \vspace{0.3cm}

      \textbf{특징}
      \begin{itemize}
        \setlength{\itemsep}{0pt}
        \item 복수 진단 (Comorbidities) 지원
        \item 시간에 따른 변화 패턴 반영
        \item 생리학적 일관성 유지
      \end{itemize}
    \end{column}
  \end{columns}

  \vspace{0.3cm}

  \begin{beamercolorbox}[sep=0.4em,wd=\textwidth,rounded=true]{block title}
    \small \textbf{생성 비용:} 환자 1명당 \textbf{\$0.12} (Claude Sonnet 4.5 기준)
    \hfill {\footnotesize 1,000명 $\approx$ \$120 / 10,000명 $\approx$ \$1,200}
  \end{beamercolorbox}
\end{frame}

% ============================================
% Generation Process
% ============================================
\begin{frame}{생성 프로세스}
  \begin{center}
    \begin{tikzpicture}[
      node distance=0.8cm,
      box/.style={rectangle, rounded corners, minimum width=9cm, text centered, draw=black, fill=gray!10, font=\ttfamily\small, align=left, inner sep=0.5cm}
    ]

    \node (code) [box] {
      \textbf{Input:}  ICD-9 codes + Descriptions\\
      \hspace{1.5cm}(e.g., CHF + Diabetes + CAD)\\[0.3cm]
      \textbf{Claude:} 1. D\_ITEMS.csv 탐색 (ITEMID 확인)\\
      \hspace{1.5cm}2. 질병의 병태생리 분석\\
      \hspace{1.5cm}3. Comorbidity 상호작용 고려\\
      \hspace{1.5cm}4. Timeline 기반 데이터 생성\\[0.3cm]
      \textbf{Output:} CHARTEVENTS.csv + LABEVENTS.csv
    };

    \end{tikzpicture}
  \end{center}

  \vspace{0.5cm}

  \textcolor{gray}{\textbf{예시 Comorbidity 조합:}}
  \begin{itemize}
    \small
    \item CHF + CAD + Diabetes (3개 동시)
    \item AKI + CHF + Anemia (3개 동시)
    \item CAD + Afib + HTN + Hyperlipidemia (4개 동시)
  \end{itemize}
\end{frame}

% ============================================
% Current Status
% ============================================
\begin{frame}{현재 진행 상황}
  \begin{itemize}
    \item[\textcolor{completed}{\checkmark}] Synthetic data 생성 파이프라인 구축 완료
    \item[\textcolor{completed}{\checkmark}] Claude Agent SDK 기반 자동화 구현
    \item[\textcolor{completed}{\checkmark}] 다양한 Comorbidity 조합 테스트
    \item[\textcolor{completed}{\checkmark}] Downstream task (Mortality Prediction) 초기 검증 (N=10)
    \item[\textcolor{inprogress}{$\rightarrow$}] \textbf{생성 스케일업 필요} (10명 $\rightarrow$ 1,000명+)
  \end{itemize}

  \vspace{0.8cm}

  \begin{beamercolorbox}[sep=0.8em,wd=\textwidth,rounded=true]{block title}
    \textbf{Next Step:} 대규모 생성 후 체계적 평가 진행 (현재 N=10으로는 통계 검증 불가)
  \end{beamercolorbox}
\end{frame}

% ============================================
% Validation Results
% ============================================
\begin{frame}{초기 검증: Mortality Prediction 결과}
  \begin{columns}[T]
    \begin{column}{0.55\textwidth}
      \begin{center}
        \footnotesize
        \begin{tabular}{c c c c c}
          \toprule
          \textbf{Patient} & \textbf{GT} & \textbf{Pred} & \textbf{Confidence} & \\
          \midrule
          90010 & SURV & SURV & 0.342 & \textcolor{completed}{\checkmark} \\
          90011 & SURV & SURV & 0.069 & \textcolor{completed}{\checkmark} \\
          90012 & DIED & DIED & 0.977 & \textcolor{completed}{\checkmark} \\
          90013 & SURV & SURV & 0.031 & \textcolor{completed}{\checkmark} \\
          90014 & SURV & DIED & 0.598 & \textcolor{red}{\texttimes} \\
          90015 & DIED & DIED & 0.934 & \textcolor{completed}{\checkmark} \\
          90016 & SURV & SURV & 0.480 & \textcolor{completed}{\checkmark} \\
          90017 & DIED & DIED & 0.702 & \textcolor{completed}{\checkmark} \\
          90018 & DIED & DIED & 0.567 & \textcolor{completed}{\checkmark} \\
          90019 & SURV & SURV & 0.039 & \textcolor{completed}{\checkmark} \\
          \bottomrule
        \end{tabular}
      \end{center}
    \end{column}
    \begin{column}{0.4\textwidth}
      \begin{center}
        \small
        \begin{tabular}{l c}
          \toprule
          \textbf{Dataset} & \textbf{Acc} \\
          \midrule
          Original MIMIC & 88.45\% \\
          \textbf{Synthetic} & \textcolor{completed}{\textbf{90.0\%}} \\
          \bottomrule
        \end{tabular}
      \end{center}

      \vspace{0.5cm}

      \begin{beamercolorbox}[sep=0.5em,rounded=true]{block body}
        \small
        Synthetic data가\\
        \textbf{유사한 예측 성능}을 보임\\
        $\rightarrow$ 체계적 평가 필요
      \end{beamercolorbox}
    \end{column}
  \end{columns}
\end{frame}

% ============================================
% Evaluation Framework Overview
% ============================================
\begin{frame}{Synthetic Data 평가 프레임워크}
  \begin{center}
    \begin{tabular}{|c|c|}
      \hline
      \cellcolor{green!20}\textbf{1. Statistical Fidelity} & \cellcolor{orange!20}\textbf{2. Utility} \\
      \cellcolor{green!20}통계적 유사성 & \cellcolor{orange!20}ML 유용성 \\
      \hline
      \cellcolor{blue!20}\textbf{3. Privacy} & \cellcolor{purple!20}\textbf{4. Clinical Validity} \\
      \cellcolor{blue!20}프라이버시 보호 & \cellcolor{purple!20}임상적 타당성 \\
      \hline
    \end{tabular}
  \end{center}

  \vspace{0.5cm}

  \begin{columns}[T]
    \begin{column}{0.48\textwidth}
      \textbf{\textcolor{completed}{1. Statistical Fidelity}}
      \begin{itemize}
        \footnotesize
        \item 변수별 분포 비교
        \item 변수 간 상관관계 보존
        \item 시계열 패턴 유사성
      \end{itemize}
    \end{column}
    \begin{column}{0.48\textwidth}
      \textbf{\textcolor{hold}{2. Utility (ML Efficacy)}}
      \begin{itemize}
        \footnotesize
        \item TSTR: Train Synthetic, Test Real
        \item TRTS: Train Real, Test Synthetic
        \item 다양한 Downstream Tasks
      \end{itemize}
    \end{column}
  \end{columns}
\end{frame}

% ============================================
% Evaluation Methods Detail
% ============================================
\begin{frame}{평가 방법 상세}
  \begin{columns}[T]
    \begin{column}{0.48\textwidth}
      \begin{block}{Statistical Fidelity 평가}
        \footnotesize
        \begin{itemize}
          \item \textbf{분포 비교}
            \begin{itemize}
              \scriptsize
              \item KL Divergence
              \item Wasserstein Distance
              \item KS Test (Kolmogorov-Smirnov)
            \end{itemize}
          \item \textbf{상관관계}
            \begin{itemize}
              \scriptsize
              \item Correlation Matrix 비교
              \item Pair-wise 산점도
            \end{itemize}
          \item \textbf{시계열}
            \begin{itemize}
              \scriptsize
              \item Autocorrelation 비교
              \item Trend/Seasonality 패턴
            \end{itemize}
        \end{itemize}
      \end{block}
    \end{column}

    \begin{column}{0.48\textwidth}
      \begin{block}{Utility 평가 (TSTR/TRTS)}
        \footnotesize
        \begin{itemize}
          \item \textbf{Downstream Tasks}
            \begin{itemize}
              \scriptsize
              \item Mortality Prediction {\color{completed}\checkmark}
              \item Length of Stay 예측
              \item Readmission 예측
              \item Sepsis 조기 예측
            \end{itemize}
          \item \textbf{평가 지표}
            \begin{itemize}
              \scriptsize
              \item AUROC, AUPRC
              \item Sensitivity, Specificity
              \item Calibration (Brier Score)
            \end{itemize}
        \end{itemize}
      \end{block}
    \end{column}
  \end{columns}
\end{frame}

% ============================================
% Privacy & Clinical Validity
% ============================================
\begin{frame}{Privacy \& Clinical Validity 평가}
  \begin{columns}[T]
    \begin{column}{0.48\textwidth}
      \begin{block}{Privacy 평가}
        \scriptsize
        \textbf{목표:} 원본 데이터로부터의 정보 유출 방지
        \begin{itemize}
          \setlength{\itemsep}{2pt}
          \item \textbf{DCR} (Distance to Closest Record)
            {\tiny -- Synthetic $\leftrightarrow$ Real 거리 측정}
          \item \textbf{Membership Inference Attack}
            {\tiny -- 특정 환자의 학습 데이터 포함 여부 추론}
          \item \textbf{Attribute Inference Attack}
            {\tiny -- 민감 속성 추론 가능성 평가}
        \end{itemize}
      \end{block}
    \end{column}

    \begin{column}{0.48\textwidth}
      \begin{block}{Clinical Validity 평가}
        \scriptsize
        \textbf{목표:} 의학적으로 타당한 데이터인지 검증
        \begin{itemize}
          \setlength{\itemsep}{2pt}
          \item \textbf{임상 규칙 준수}
            {\tiny -- Vital sign 범위, 생리학적 상호관계}
          \item \textbf{Expert Review}
            {\tiny -- 임상의 blind review, Real vs Synthetic 구분}
          \item \textbf{Disease Progression}
            {\tiny -- 질병 자연경과, 치료 반응 패턴}
        \end{itemize}
      \end{block}
    \end{column}
  \end{columns}
\end{frame}

% ============================================
% Evaluation Roadmap
% ============================================
\begin{frame}{평가 로드맵}
  \small
  \begin{block}{\textcolor{completed}{Phase 1: 기초 통계 검증} \checkmark 진행 중}
    \footnotesize 분포 비교, Mortality Prediction 초기 결과
  \end{block}
  \vspace{-0.1cm}
  \centerline{$\Downarrow$}
  \vspace{-0.1cm}
  \begin{block}{Phase 2: Utility 확장 검증}
    \footnotesize LOS, Readmission, Sepsis 등 다양한 Task로 확대
  \end{block}
  \vspace{-0.1cm}
  \centerline{$\Downarrow$}
  \vspace{-0.1cm}
  \begin{block}{Phase 3: Privacy 검증}
    \footnotesize DCR 분석, Membership/Attribute Inference Attack 테스트
  \end{block}
  \vspace{-0.1cm}
  \centerline{$\Downarrow$}
  \vspace{-0.1cm}
  \begin{block}{Phase 4: Clinical Validity 검증}
    \footnotesize 임상 규칙 자동 검증 + Expert Review
  \end{block}
\end{frame}

% ============================================
% Summary
% ============================================
\begin{frame}{Summary \& Next Steps}
  \begin{columns}[T]
    \begin{column}{0.48\textwidth}
      \begin{block}{이번 주 완료}
        \begin{itemize}
          \item 논문 초안 2건 전달
            \begin{itemize}
              \small
              \item 간호 용어 분류 모델
              \item 뇌출혈 위험 예측 모델
            \end{itemize}
          \item ARDS 데이터 전달
          \item 사망 데이터 추출 완료
          \item DEEP-ICU
            \begin{itemize}
              \small
              \item Synthetic data 생성 파이프라인 구축
              \item Mortality Prediction 초기 검증
            \end{itemize}
        \end{itemize}
      \end{block}
    \end{column}

    \begin{column}{0.48\textwidth}
      \begin{block}{다음 주 계획}
        \begin{itemize}
          \item DEEP-ICU 생성 스케일업
            \begin{itemize}
              \small
              \item 현재 10명 $\rightarrow$ 1,000명+ 목표
              \item 병렬 생성 파이프라인 구축
            \end{itemize}
          \item 생성 후 평가 진행
          \item SNOMED CT \& LOINC mapping 연구 계획 구체화
          \item 논문 피드백 반영
        \end{itemize}
      \end{block}
    \end{column}
  \end{columns}
\end{frame}

% ============================================
% Q&A
% ============================================
\begin{frame}
  \begin{center}
    \vspace{2cm}
    {\Huge \textbf{Q \& A}}

    \vspace{2cm}

    {\Large 감사합니다}
  \end{center}
\end{frame}

\end{document}
