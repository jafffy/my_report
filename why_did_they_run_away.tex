\documentclass[aspectratio=169]{beamer}

% Theme
\usetheme{Madrid}
\usecolortheme{whale}

% Packages
\usepackage{kotex}
\usepackage{graphicx}
\usepackage{booktabs}
\usepackage{xcolor}
\usepackage{colortbl}
\usepackage{tikz}
\usetikzlibrary{shapes.geometric, arrows, positioning}

% Colors
\definecolor{completed}{RGB}{34, 139, 34}
\definecolor{inprogress}{RGB}{70, 130, 180}
\definecolor{hold}{RGB}{255, 140, 0}
\definecolor{highlight}{RGB}{128, 0, 128}
\definecolor{emergency}{RGB}{220, 53, 69}
\definecolor{mental}{RGB}{147, 112, 219}
\definecolor{alcohol}{RGB}{255, 193, 7}

% Remove navigation symbols
\setbeamertemplate{navigation symbols}{}

% Footer
\setbeamertemplate{footline}{
  \leavevmode%
  \hbox{%
    \begin{beamercolorbox}[wd=.333333\paperwidth,ht=2.25ex,dp=1ex,center]{author in head/foot}%
      \usebeamerfont{author in head/foot}K-MIMIC Analysis
    \end{beamercolorbox}%
    \begin{beamercolorbox}[wd=.333333\paperwidth,ht=2.25ex,dp=1ex,center]{title in head/foot}%
      \usebeamerfont{title in head/foot}Run Away 퇴원 분석
    \end{beamercolorbox}%
    \begin{beamercolorbox}[wd=.333333\paperwidth,ht=2.25ex,dp=1ex,right]{date in head/foot}%
      \usebeamerfont{date in head/foot}2025.12.26\hspace*{2em}
      \insertframenumber{} / \inserttotalframenumber\hspace*{2ex}
    \end{beamercolorbox}}%
  \vskip0pt%
}

% Title information
\title[Run Away 분석]{K-MIMIC ``Run Away'' 퇴원 분석 보고서}
\author{최재원}
\date{2025년 12월 26일}
\institute{서울대학교병원 융합의학과}

\begin{document}

% ============================================
% Title Slide
% ============================================
\begin{frame}
  \titlepage
\end{frame}

% ============================================
% Overview Slide
% ============================================
\begin{frame}{개요}
  \begin{center}
    \begin{tabular}{l c}
      \toprule
      \textbf{항목} & \textbf{수치} \\
      \midrule
      총 Run away 케이스 & \textbf{409건} \\
      관련 병원 수 & 8개 병원 \\
      가장 많은 병원 & 병원 201 (356건, 87\%) \\
      \bottomrule
    \end{tabular}
  \end{center}

  \vspace{1cm}

  \begin{beamercolorbox}[sep=0.8em,wd=\textwidth,rounded=true,shadow=true]{block body}
    \centering
    \large 병원 201에서 전체 Run away의 \textbf{87\%}가 발생
  \end{beamercolorbox}
\end{frame}

% ============================================
% Patient Characteristics
% ============================================
\begin{frame}{Run away 환자 기본 특성}
  \begin{columns}[T]
    \begin{column}{0.48\textwidth}
      \begin{block}{입원 유형}
        \small
        \begin{tabular}{l c c}
          \toprule
          \textbf{유형} & \textbf{건수} & \textbf{비율} \\
          \midrule
          \textcolor{emergency}{\textbf{응급실}} & \textbf{351건} & \textbf{86\%} \\
          일반입원 & 50건 & 12\% \\
          외래 & 5건 & 1\% \\
          \bottomrule
        \end{tabular}
      \end{block}
    \end{column}

    \begin{column}{0.48\textwidth}
      \begin{block}{보험 유형}
        \small
        \begin{tabular}{l c c}
          \toprule
          \textbf{보험} & \textbf{건수} & \textbf{비율} \\
          \midrule
          국민건강보험 & 314건 & 77\% \\
          medical ins. 1 & 76건 & 19\% \\
          무보험/기타 & 19건 & 4\% \\
          \bottomrule
        \end{tabular}
      \end{block}
    \end{column}
  \end{columns}

  \vspace{0.5cm}

  \begin{beamercolorbox}[sep=0.5em,wd=\textwidth,rounded=true]{block title}
    \centering
    \textbf{86\%}가 \textcolor{emergency}{응급실} 경유 입원 $\rightarrow$ 비자발적 내원 후 이탈 가능성
  \end{beamercolorbox}
\end{frame}

% ============================================
% Main Diagnosis Categories
% ============================================
\begin{frame}{핵심 발견: 병원 201 주진단 대분류 (114건 분석)}
  \begin{center}
    \small
    \begin{tabular}{c l c l}
      \toprule
      \textbf{ICD-10} & \textbf{분류} & \textbf{건수} & \textbf{주요 질환} \\
      \midrule
      \textcolor{mental}{\textbf{F}} & \textcolor{mental}{\textbf{정신/행동 장애}} & \textcolor{mental}{\textbf{16건}} & 우울증, 알코올의존, 조현병 \\
      K & 소화기계 & 13건 & \textbf{알코올성 간질환} 포함 \\
      \textcolor{emergency}{\textbf{T}} & \textcolor{emergency}{\textbf{중독/외인}} & \textcolor{emergency}{\textbf{11건}} & 약물중독, 자해 의심 \\
      S & 손상 & 10건 & 외상 \\
      I & 순환기계 & 8건 & 심혈관 질환 \\
      N & 비뇨생식계 & 8건 & 신장질환 \\
      \bottomrule
    \end{tabular}
  \end{center}

  \vspace{0.5cm}

  \begin{columns}[T]
    \begin{column}{0.48\textwidth}
      \begin{beamercolorbox}[sep=0.4em,rounded=true]{block body}
        \centering
        \textcolor{mental}{\textbf{F코드 (정신과)}} = 16건
      \end{beamercolorbox}
    \end{column}
    \begin{column}{0.48\textwidth}
      \begin{beamercolorbox}[sep=0.4em,rounded=true]{block body}
        \centering
        \textcolor{emergency}{\textbf{T코드 (중독/자해)}} = 11건
      \end{beamercolorbox}
    \end{column}
  \end{columns}
\end{frame}

% ============================================
% Psychiatric Diagnosis Detail
% ============================================
\begin{frame}{정신과 진단 상세 (F코드)}
  \begin{columns}[T]
    \begin{column}{0.55\textwidth}
      \begin{block}{F코드 상세 분류}
        \small
        \begin{tabular}{l l c}
          \toprule
          \textbf{코드} & \textbf{진단명} & \textbf{건수} \\
          \midrule
          F102 & \textcolor{alcohol}{\textbf{알코올 의존 증후군}} & 4건 \\
          F321, F329 & \textcolor{mental}{\textbf{우울 에피소드}} & 4건 \\
          F200, F209 & 조현병 & 2건 \\
          F411, F419 & 불안장애 & 2건 \\
          F101, F104 & 알코올 남용/금단 & 2건 \\
          F603 & 경계성 인격장애 & 1건 \\
          \bottomrule
        \end{tabular}
      \end{block}
    \end{column}

    \begin{column}{0.42\textwidth}
      \begin{block}{알코올성 간질환 (K코드)}
        \small
        \begin{tabular}{l c}
          \toprule
          \textbf{진단명} & \textbf{건수} \\
          \midrule
          알코올성 간섬유증/경변 & 2건 \\
          간부전 & 2건 \\
          \bottomrule
        \end{tabular}
      \end{block}

      \vspace{0.3cm}

      \begin{beamercolorbox}[sep=0.4em,rounded=true]{block title}
        \footnotesize
        \textbf{알코올 관련} 진단이\\
        전체의 상당수를 차지
      \end{beamercolorbox}
    \end{column}
  \end{columns}
\end{frame}

% ============================================
% Poisoning/Self-harm
% ============================================
\begin{frame}{중독/자해 관련 (T코드)}
  \begin{center}
    \begin{tabular}{l l c}
      \toprule
      \textbf{코드} & \textbf{진단명} & \textbf{건수} \\
      \midrule
      \textcolor{emergency}{\textbf{T659}} & \textcolor{emergency}{\textbf{상세불명 물질 독작용 (자해의심)}} & \textcolor{emergency}{\textbf{6건}} \\
      T658 & 기타 물질 독작용 & 1건 \\
      \bottomrule
    \end{tabular}
  \end{center}

  \vspace{1cm}

  \begin{beamercolorbox}[sep=1em,wd=\textwidth,rounded=true,shadow=true]{block body}
    \centering
    \large
    T659 (물질 독작용)가 다수 $\rightarrow$ \\
    \textbf{자해 시도 후 응급실 내원 $\rightarrow$ 이탈 패턴} 의심
  \end{beamercolorbox}
\end{frame}

% ============================================
% Mental Health Summary
% ============================================
\begin{frame}{정신건강/알코올 관련 환자 비율 (병원 201)}
  \begin{center}
    \begin{tabular}{l c}
      \toprule
      \textbf{분류} & \textbf{환자 수} \\
      \midrule
      \textcolor{mental}{\textbf{F코드(정신과) 진단 환자}} & \textcolor{mental}{\textbf{22명}} \\
      \textcolor{alcohol}{\textbf{알코올성 간질환(K70) 환자}} & \textcolor{alcohol}{\textbf{6명}} \\
      \textbf{중복 제외 추정} & \textbf{$\sim$25명 이상} \\
      \bottomrule
    \end{tabular}
  \end{center}

  \vspace{1cm}

  \begin{beamercolorbox}[sep=1em,wd=\textwidth,rounded=true,shadow=true]{block title}
    \centering
    \Large
    진단 기록이 있는 환자 중 \\
    \textbf{약 20-25\%가 정신건강 또는 알코올 관련 문제}
  \end{beamercolorbox}
\end{frame}

% ============================================
% Hospital Comparison
% ============================================
\begin{frame}{병원별 비교}
  \begin{center}
    \begin{tabular}{l c l}
      \toprule
      \textbf{병원} & \textbf{Run away 수} & \textbf{주요 진단} \\
      \midrule
      \textcolor{emergency}{\textbf{201}} & \textcolor{emergency}{\textbf{356건}} & F(정신과), K(간질환), T(중독) \\
      006 & 24건 & S(손상), I(순환기) \\
      001 & 7건 & I(순환기), C(암) \\
      \bottomrule
    \end{tabular}
  \end{center}

  \vspace{0.8cm}

  \begin{beamercolorbox}[sep=0.8em,wd=\textwidth,rounded=true]{block body}
    \centering
    병원 201의 Run away 환자는 \textbf{정신과/알코올/중독} 관련 진단 비율이 높음\\
    $\rightarrow$ 병원 특성 또는 환자군의 차이 가능성
  \end{beamercolorbox}
\end{frame}

% ============================================
% Conclusion
% ============================================
\begin{frame}{결론: Run away의 주요 원인 추정}
  \begin{columns}[T]
    \begin{column}{0.48\textwidth}
      \begin{block}{\textcolor{mental}{1. 정신건강 문제 (가장 주요)}}
        \footnotesize
        \begin{itemize}
          \item 우울증, 조현병, 불안장애 환자
          \item 치료 거부 또는 충동적 퇴원
          \item 병식(insight) 부족으로 인한 자의 퇴원
        \end{itemize}
      \end{block}

      \vspace{0.3cm}

      \begin{block}{\textcolor{alcohol}{2. 알코올 관련 문제}}
        \footnotesize
        \begin{itemize}
          \item 알코올 의존/남용 환자의 금단 증상 시 이탈
          \item 알코올성 간질환 환자의 음주 욕구
        \end{itemize}
      \end{block}
    \end{column}

    \begin{column}{0.48\textwidth}
      \begin{block}{\textcolor{emergency}{3. 자해/중독 환자}}
        \footnotesize
        \begin{itemize}
          \item T659(물질 독작용)가 다수
          \item 자해 시도 후 응급실 내원 $\rightarrow$ 이탈 패턴
        \end{itemize}
      \end{block}

      \vspace{0.3cm}

      \begin{block}{4. 응급실 특성}
        \footnotesize
        \begin{itemize}
          \item 86\%가 응급실 입원
          \item 비자발적 내원 후 이탈 가능성
          \item 대기 시간, 강제 입원 회피 등
        \end{itemize}
      \end{block}
    \end{column}
  \end{columns}
\end{frame}

% ============================================
% Q&A
% ============================================
\begin{frame}
  \begin{center}
    \vspace{2cm}
    {\Huge \textbf{Q \& A}}

    \vspace{2cm}

    {\Large 감사합니다}
  \end{center}
\end{frame}

\end{document}
