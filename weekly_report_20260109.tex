\documentclass[aspectratio=169]{beamer}

% Theme
\usetheme{Madrid}
\usecolortheme{whale}

% Packages
\usepackage{kotex}
\usepackage{graphicx}
\usepackage{booktabs}
\usepackage{xcolor}
\usepackage{colortbl}
\usepackage{tikz}
\usetikzlibrary{shapes.geometric, arrows, positioning}

% Colors
\definecolor{completed}{RGB}{34, 139, 34}
\definecolor{inprogress}{RGB}{70, 130, 180}
\definecolor{hold}{RGB}{255, 140, 0}
\definecolor{highlight}{RGB}{128, 0, 128}

% Remove navigation symbols
\setbeamertemplate{navigation symbols}{}

% Footer
\setbeamertemplate{footline}{
  \leavevmode%
  \hbox{%
    \begin{beamercolorbox}[wd=.333333\paperwidth,ht=2.25ex,dp=1ex,center]{author in head/foot}%
      \usebeamerfont{author in head/foot}최재원
    \end{beamercolorbox}%
    \begin{beamercolorbox}[wd=.333333\paperwidth,ht=2.25ex,dp=1ex,center]{title in head/foot}%
      \usebeamerfont{title in head/foot}주간 업무 보고
    \end{beamercolorbox}%
    \begin{beamercolorbox}[wd=.333333\paperwidth,ht=2.25ex,dp=1ex,right]{date in head/foot}%
      \usebeamerfont{date in head/foot}2026.01.09\hspace*{2em}
      \insertframenumber{} / \inserttotalframenumber\hspace*{2ex}
    \end{beamercolorbox}}%
  \vskip0pt%
}

% Title information
\title[주간 보고]{주간 업무 보고}
\author{최재원}
\date{2026년 1월 9일}
\institute{서울대학교병원 융합의학과}

\begin{document}

% ============================================
% Title Slide
% ============================================
\begin{frame}
  \titlepage
\end{frame}

% ============================================
% Overview Slide
% ============================================
\begin{frame}{업무 현황 Overview}
  \begin{columns}[T]
    \begin{column}{0.48\textwidth}
      \textbf{\textcolor{completed}{완료 (Completed)}}
      \vspace{0.2cm}

      \footnotesize
      \begin{tabular}{p{4.5cm} l}
        \toprule
        \textbf{프로젝트} & \textbf{상태} \\
        \midrule
        SNOMED CT 간호 용어 분류 & 논문 초안 전달 \\
        사망 데이터 활용 연구 & 데이터 추출 \\
        뇌조직 생검술 뇌출혈 예측 & 논문 초안 전달 \\
        ARDS 연구 (김성민 교수님) & 데이터 전달 \\
        \bottomrule
      \end{tabular}
    \end{column}

    \begin{column}{0.48\textwidth}
      \textbf{\textcolor{inprogress}{진행 중 (In Progress)}}
      \vspace{0.2cm}

      \footnotesize
      \begin{tabular}{p{4.5cm} l}
        \toprule
        \textbf{프로젝트} & \textbf{상태} \\
        \midrule
        의료 프로그램 자동화 & 개발 중 \\
        SNOMED \& LOINC & Validation \\
        DEEP-ICU Synthetic Data & Local LLM 테스트 \\
        K-MIMIC 도주 환자 & ITEMID 검색 \\
        K-MIMIC 문제점 정리 & 리스트 작성 \\
        자폐 플랫폼 (김동영) & 논문 writing \\
        \bottomrule
      \end{tabular}
    \end{column}
  \end{columns}
\end{frame}

% ============================================
% 의료 프로그램 자동화 Title
% ============================================
\begin{frame}
  \begin{center}
    \vspace{2cm}
    {\Huge \textbf{의료 프로그램 업무 자동화}}

    \vspace{0.5cm}

    {\large 사구체 최적 Zoom 결정을 위한\\RL Agent 개발}

    \vspace{1cm}

    {\color{gray} Reinforcement Learning 기반 접근}
  \end{center}
\end{frame}

% ============================================
% RL Agent 개발
% ============================================
\begin{frame}{RL Agent: 사구체 최적 Zoom 결정}
  \begin{columns}[T]
    \begin{column}{0.48\textwidth}
      \textbf{문제 정의}
      \begin{itemize}
        \item 병리 이미지에서 사구체 관찰 시 최적의 zoom level 자동 결정
        \item 수동 조작의 반복적 작업 자동화
        \item 일관된 관찰 품질 확보
      \end{itemize}

      \vspace{0.5cm}

      \textbf{RL 접근 방식}
      \begin{itemize}
        \item State: 현재 이미지 상태
        \item Action: Zoom in/out/maintain
        \item Reward: 사구체 가시성 품질
      \end{itemize}
    \end{column}

    \begin{column}{0.48\textwidth}
      \textbf{현재 진행 상황}
      \begin{itemize}
        \item[\textcolor{inprogress}{$\rightarrow$}] RL Agent 아키텍처 설계 중
        \item[\textcolor{inprogress}{$\rightarrow$}] 학습 환경 구축 중
      \end{itemize}

      \vspace{0.5cm}

      \textbf{기대 효과}
      \begin{itemize}
        \item 업무 효율성 향상
        \item 관찰 품질 표준화
        \item 반복 작업 시간 절감
      \end{itemize}
    \end{column}
  \end{columns}
\end{frame}

% ============================================
% SNOMED CT & LOINC Title
% ============================================
\begin{frame}
  \begin{center}
    \vspace{2cm}
    {\Huge \textbf{SNOMED CT \& LOINC Mapping}}

    \vspace{0.5cm}

    {\large Human-AI Collaborative Approach}

    \vspace{1cm}

    {\color{gray} KLUE-BERT 기반 LOINC Mapping 모델}
  \end{center}
\end{frame}

% ============================================
% SNOMED CT & LOINC Mapping Results
% ============================================
\begin{frame}{SNOMED CT \& LOINC Mapping: Validation 결과}
  \begin{columns}[T]
    \begin{column}{0.52\textwidth}
      \begin{center}
        \textbf{모델 성능 비교 (KLUE-BERT)}

        \vspace{0.3cm}

        \footnotesize
        \begin{tabular}{l c c}
          \toprule
          & \textbf{SNOMED CT} & \textbf{LOINC} \\
          \midrule
          Classes & 2,079 & 391 \\
          Test Samples & 6,793 & 539 \\
          \midrule
          \textbf{Top-1 Acc} & 75.40\% & 65.0\% \\
          \textbf{Top-5 Acc} & 81.54\% & 79.3\% \\
          \bottomrule
        \end{tabular}
      \end{center}

      \vspace{0.3cm}

      \begin{beamercolorbox}[sep=0.4em,rounded=true]{block body}
        \scriptsize
        \textbf{SNOMED CT가 더 높은 이유:}\\
        간호사 선생님들 의견 -- 문장에 정보량이\\
        더 많아서 맞추기 쉬웠을 것
      \end{beamercolorbox}
    \end{column}

    \begin{column}{0.45\textwidth}
      \textbf{결과 분석}
      \begin{itemize}
        \footnotesize
        \item SNOMED CT: 75\%도 실용성 부족
        \item LOINC: Top-1 65\%는 더 낮음
        \item 단독 BERT 모델의 한계
      \end{itemize}

      \vspace{0.4cm}

      \textbf{향후 개선 방향}
      \begin{itemize}
        \footnotesize
        \item LLM 후처리 추가 (ACE approach 적용)
      \end{itemize}
    \end{column}
  \end{columns}
\end{frame}

% ============================================
% DEEP-ICU Title
% ============================================
\begin{frame}
  \begin{center}
    \vspace{2cm}
    {\Huge \textbf{DEEP-ICU}}

    \vspace{0.5cm}

    {\large Domain-Embedded Encodings for\\Personalized ICU outcomes}

    \vspace{1cm}

    {\color{gray} K-MIMIC Synthetic Data Generation}
  \end{center}
\end{frame}

% ============================================
% DEEP-ICU Local LLM
% ============================================
\begin{frame}{DEEP-ICU: Local LLM 생성 테스트}
  \begin{columns}[T]
    \begin{column}{0.48\textwidth}
      \textbf{변경 사항}
      \begin{itemize}
        \item 기존: Claude Sonnet 4.5 (API)
        \item 현재: Local LLM 테스트
      \end{itemize}

      \vspace{0.5cm}

      \textbf{Local LLM}
      \begin{itemize}
        \item 비용 절감 (API 비용 없음)
        \item 데이터 외부 전송 없음
        \item 대규모 생성 시 경제적
      \end{itemize}
    \end{column}

    \begin{column}{0.48\textwidth}
      \textbf{현재 진행 상황}
      \begin{itemize}
        \item[\textcolor{inprogress}{$\rightarrow$}] Local LLM 환경 구축
        \item[\textcolor{inprogress}{$\rightarrow$}] 생성 품질 테스트 중
        \item[\textcolor{inprogress}{$\rightarrow$}] Claude 대비 품질 비교
      \end{itemize}

      \vspace{0.5cm}

      \textbf{확인 필요 사항}
      \begin{itemize}
        \item 생성 품질 비교
        \item 생성 속도 측정
        \item 임상적 타당성 검증
      \end{itemize}
    \end{column}
  \end{columns}

  \vspace{0.3cm}

  \begin{beamercolorbox}[sep=0.4em,wd=\textwidth,rounded=true]{block title}
    \small \textbf{목표:} Local LLM으로 동등한 품질의 Synthetic Data 생성 가능 여부 검증
  \end{beamercolorbox}
\end{frame}

% ============================================
% Summary
% ============================================
\begin{frame}{Summary \& Next Steps}
  \begin{columns}[T]
    \begin{column}{0.48\textwidth}
      \begin{block}{이번 주 진행 사항}
        \begin{itemize}
          \item 의료 프로그램 자동화
            \begin{itemize}
              \small
              \item 사구체 최적 zoom RL agent 개발 착수
            \end{itemize}
          \item SNOMED CT \& LOINC Mapping
            \begin{itemize}
              \small
              \item SNOMED: Top-1 75.4\%, Top-5 81.5\%
              \item LOINC: Top-1 65.0\%, Top-5 79.3\%
            \end{itemize}
          \item DEEP-ICU
            \begin{itemize}
              \small
              \item Local LLM 생성 테스트 진행
            \end{itemize}
          \item K-MIMIC 도주 환자 분석
            \begin{itemize}
              \small
              \item 도주 관련 ITEMID mapping 검색
            \end{itemize}
        \end{itemize}
      \end{block}
    \end{column}

    \begin{column}{0.48\textwidth}
      \begin{block}{다음 주 계획}
        \begin{itemize}
          \item RL Agent 개발 지속
            \begin{itemize}
              \small
              \item 학습 환경 완성
              \item 초기 학습 실험
            \end{itemize}
          \item LOINC Mapping 개선
            \begin{itemize}
              \small
              \item LLM 후처리로 정확도 향상 시도
              \item BERT + LLM 파이프라인 구축
            \end{itemize}
          \item DEEP-ICU Local LLM
            \begin{itemize}
              \small
              \item 품질 비교 완료
              \item 대규모 생성 파이프라인
            \end{itemize}
          \item K-MIMIC 문제점 정리
            \begin{itemize}
              \small
              \item APACHE score, 기호 문제
              \item 임상관찰/간호기록 모호함
            \end{itemize}
        \end{itemize}
      \end{block}
    \end{column}
  \end{columns}
\end{frame}

% ============================================
% Q&A
% ============================================
\begin{frame}
  \begin{center}
    \vspace{2cm}
    {\Huge \textbf{Q \& A}}

    \vspace{2cm}

    {\Large 감사합니다}
  \end{center}
\end{frame}

\end{document}
